\UseRawInputEncoding


% Pacchetti per lingua e codifica
\usepackage[italian]{babel}    % Supporto per la lingua italiana
\usepackage[utf8]{inputenc}    % Supporto per i caratteri UTF-8
\usepackage[T1]{fontenc}       % Codifica dei font


% Pacchetti matematici
\usepackage{amsmath}           % Per formule matematiche
\usepackage{amssymb}           % Simboli matematici aggiuntivi

% Pacchetti per layout e struttura
\usepackage[margin=1.5cm]{geometry} % Margini personalizzati
\usepackage{fancyhdr}          % Per intestazioni e piè di pagina
\usepackage{titlesec}          % Personalizzazione titoli
\usepackage{enumitem}          % Per personalizzare elenchi
\usepackage{float}             % Posizionamento preciso
\usepackage{tabularx}          % Tabelle avanzate
\usepackage{nameref}           % Riferimenti per nome
\usepackage{pdfpages}          % Inclusione di PDF
\usepackage{tcolorbox}         % Per creare box colorati
\tcbuselibrary{breakable,skins} % Estensioni per tcolorbox

% Pacchetti grafici
\usepackage{graphicx}          % Per includere immagini
\usepackage{tikz}              % Disegni vettoriali
\usetikzlibrary{shapes,arrows,positioning,shadows} % Estensioni per tikz
\usepackage{xcolor}            % Gestione avanzata dei colori

% Pacchetti per il codice
\usepackage{listings}          % Per inserire blocchi di codice

% Hyperref (caricato alla fine per evitare conflitti)
\usepackage{hyperref}          % Per link ipertestuali



\usepackage{booktabs}     % Per linee orizzontali professionali
\usepackage{siunitx}
\usepackage{colortbl}     % Per colorare le celle
\usepackage{array}
\usepackage{caption} 
\captionsetup[table]{font=bf, labelfont=bf}  % Bold caption
\renewcommand{\arraystretch}{1.4}  % Increase row spacing for readability\textbf{}
\usepackage{ragged2e} % Per un migliore allineamento del testo nelle celle (es. \RaggedRight)

% Configurazioni di pagina
\pagestyle{plain}              

% Personalizzazione del formato dei capitoli
\titleformat{\chapter}[display]
{\normalfont\huge\bfseries\color{blue!70!black}}
{\chaptertitlename\ \thechapter}{20pt}{\Huge}
[\vspace{2ex}\hrule\vspace{1ex}]

% Personalizzazione delle sezioni
\titleformat{\section}
{\normalfont\Large\bfseries\color{blue!60!black}}
{\thesection}{1em}{}
[\vspace{1ex}\hrule height 0.5pt\vspace{1ex}]

% Personalizzazione delle sottosezioni
\titleformat{\subsection}
{\normalfont\large\bfseries\color{blue!50!black}}
{\thesubsection}{1em}{}

% Definizione di stili per tikz
\tikzstyle{variable}=[draw, fill=blue!20, minimum width=2cm, minimum height=0.8cm, font=\bfseries, text centered]
\tikzstyle{memory}=[draw, fill=green!20, minimum width=3.5cm, minimum height=1.2cm, rounded corners, font=\bfseries, align=center]
\tikzstyle{reference}=[->, thick, blue]
\tikzstyle{label}=[font=\small\bfseries]
\tikzstyle{section title}=[font=\bfseries\large, text centered, text width=12cm]
\tikzstyle{subsection title}=[font=\bfseries, text centered, text width=10cm]
\tikzstyle{explanation}=[text width=12cm, align=center, font=\small]
\tikzstyle{id label}=[font=\small, text centered]

% Definizione colori per codice
\definecolor{codegreen}{rgb}{0,0.6,0}
\definecolor{codegray}{rgb}{0.5,0.5,0.5}
\definecolor{codepurple}{rgb}{0.58,0,0.82}
\definecolor{backcolour}{rgb}{0.95,0.95,0.95}

% Stile per codice Python
\lstdefinestyle{pythonstyle}{
    backgroundcolor=\color{backcolour},   
    commentstyle=\color{codegreen},
    keywordstyle=\color{orange},
    numberstyle=\tiny\color{codegray},
    stringstyle=\color{codepurple},
    basicstyle=\ttfamily\footnotesize,
    breakatwhitespace=true,         
    breaklines=true,                 
    captionpos=b,                    
    keepspaces=true,                 
    numbers=left,                    
    numbersep=5pt,                  
    showspaces=false,                
    showstringspaces=false,
    showtabs=false,                  
    tabsize=2,
    language=Python,
    xleftmargin=0.04\textwidth,
    xrightmargin=0.04\textwidth,
    linewidth=1.1\textwidth,
    postbreak=\mbox{\textcolor{red}{$\hookrightarrow$}\space},
    breakindent=0pt,
    columns=flexible,
    commentstyle=\color{codegreen}\raggedright,
}


% Definizione dei colori per una migliore leggibilità
\definecolor{setA}{RGB}{255,182,193}      % Rosa chiaro per insieme A
\definecolor{setB}{RGB}{173,216,230}      % Azzurro chiaro per insieme B  
\definecolor{intersection}{RGB}{221,160,221} % Viola chiaro per intersezione
\definecolor{union}{RGB}{255,228,196}     % Pesca per unione completa


% Configurazione unificata per listings
\lstset{style=pythonstyle}
\lstset{
breaklines=true,             % Permetti il ritorno a capo per il codice
breakatwhitespace=true,      % Ritorna a capo solo agli spazi bianchi
style=pythonstyle,
postbreak=\mbox{\textcolor{red}{$\hookrightarrow$}\space}, % Indicatore di continuazione
breakindent=0pt,             % Nessuna indentazione dopo il ritorno a capo
columns=flexible,            % Spaziatura flessibile
keepspaces=true,             % Mantieni gli spazi
commentstyle=\color{codegreen}\raggedright, % Impedisci il ritorno a capo nei commenti
inputencoding=utf8,          % Specifica l'encoding di input
extendedchars=true,          % Supporto per caratteri estesi
literate={à}{{`a}}1         % Sostituisce 'à' con il comando LaTeX equivalente
{è}{{e'}}1         % Sostituisce 'è' con il comando LaTeX equivalente
{ì}{{`\i}}1        % Sostituisce 'ì' con il comando LaTeX equivalente
{ò}{{o'}}1         % Sostituisce 'ò' con il comando LaTeX equivalente
{ù}{{u'}}1         % Sostituisce 'ù' con il comando LaTeX equivalente
}

% Ambienti personalizzati
\newenvironment{chapterintro}
{\begin{tcolorbox}[
    enhanced,
    colback=blue!5!white,
    colframe=blue!75!black,
    arc=0mm,
    boxrule=0.5mm,
    left=5mm,
    right=5mm,
    top=5mm,
    bottom=5mm
]}
{\end{tcolorbox}}

\newenvironment{esempio}{\begin{tcolorbox}[colback=blue!5!white,colframe=blue!75!black,title=Esempio]}{\end{tcolorbox}}

\newtcolorbox{nota}[1][]{
    breakable,
    title=Nota,
    colback=blue!5,
    colframe=blue!75!black,
    fonttitle=\bfseries,
    #1
}

\newtcolorbox{ese}[1][]{
    breakable,
    title=Esempio,
    colback=green!5,
    colframe=green!75!black,
    fonttitle=\bfseries,
    #1
}

\newtcolorbox{attenzione}[1][]{
    breakable,
    title=Attenzione,
    colback=red!5,
    colframe=red!75!black,
    fonttitle=\bfseries,
    #1
}

% Comandi personalizzati
\newcommand{\sectionsep}{
    \begin{center}
        \vspace{0.5cm}
        \textcolor{blue!70!black}{$\ast$ $\ast$ $\ast$}
        \vspace{0.5cm}
    \end{center}
}

\newcommand{\percorsoApprendimento}[2]{
\begin{center}
\begin{tikzpicture}
    \node[draw=blue!70!black, fill=blue!5, rounded corners, text width=0.85\textwidth, inner sep=10pt, drop shadow] {
        \textbf{Percorso di apprendimento:} #1
        
        \vspace{0.3cm}
        #2
    };
    % Aggiungi un indicatore grafico di "continua"
    \node[right=0.1cm of current bounding box.east, font=\Large\bfseries, text=blue!70!black] {→};
\end{tikzpicture}
\end{center}
}

