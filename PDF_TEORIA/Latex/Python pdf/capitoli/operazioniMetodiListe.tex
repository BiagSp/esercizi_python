\clearpage

\subsubsection{Operazioni e Metodi Base}

Le liste in Python offrono numerosi metodi incorporati che permettono di manipolarle in modo efficiente. Questi metodi sono funzionalità già pronte che consentono di aggiungere, rimuovere, ordinare e modificare gli elementi di una lista senza dover scrivere codice complesso.

\subsubsection{Metodi per Aggiungere Elementi}\label{MetodiListe}

Python fornisce diversi metodi per aggiungere elementi a una lista:

\begin{lstlisting}[language=Python]
# Creazione di una lista di base
frutti = ["mela", "banana"]

# Metodo append() - aggiunge un elemento alla fine della lista
frutti.append("arancia")
# Ora frutti è ["mela", "banana", "arancia"]

# Metodo insert() - inserisce un elemento in una posizione specifica
frutti.insert(1, "pera")  # Inserisce "pera" all'indice 1
# Ora frutti è ["mela", "pera", "banana", "arancia"]

# Metodo extend() - aggiunge tutti gli elementi di un'altra lista
altri_frutti = ["kiwi", "ananas"]
frutti.extend(altri_frutti)
# Ora frutti è ["mela", "pera", "banana", "arancia", "kiwi", "ananas"]
\end{lstlisting}

La differenza tra \texttt{append()} e \texttt{extend()} è importante:
\begin{itemize}
    \item \texttt{append(x)} aggiunge \texttt{x} come un singolo elemento, anche se \texttt{x} è una lista
    \item \texttt{extend(lista)} aggiunge ogni singolo elemento di \texttt{lista} alla lista originale
\end{itemize}

Esempio:
\begin{lstlisting}[language=Python]
lista1 = [1, 2, 3]
lista2 = [4, 5]

lista1.append(lista2)  # lista1 diventa [1, 2, 3, [4, 5]]
                       # La lista2 è aggiunta come unico elemento

lista3 = [1, 2, 3]
lista3.extend(lista2)  # lista3 diventa [1, 2, 3, 4, 5]
                       # Gli elementi di lista2 sono aggiunti singolarmente
\end{lstlisting}

\subsubsection{Metodi per Rimuovere Elementi}\label{MetodiListeRemove}

Per rimuovere elementi da una lista, Python offre vari metodi:

\begin{lstlisting}[language=Python]
numeri = [10, 20, 30, 40, 50, 30]

# Metodo remove() - rimuove la prima occorrenza di un valore
numeri.remove(30)  # Rimuove il primo 30 trovato
# Ora numeri è [10, 20, 40, 50, 30]

# Metodo pop() - rimuove e restituisce l'elemento all'indice specificato
elemento = numeri.pop(1)  # Rimuove e restituisce il secondo elemento (20)
print(elemento)  # 20
# Ora numeri è [10, 40, 50, 30]

# Metodo pop() senza argomenti - rimuove e restituisce l'ultimo elemento
ultimo = numeri.pop()  # Rimuove e restituisce l'ultimo elemento (30)
print(ultimo)  # 30
# Ora numeri è [10, 40, 50]

# Istruzione del - rimuove l'elemento o la sezione specificata
del numeri[0]  # Rimuove il primo elemento
# Ora numeri è [40, 50]

# Metodo clear() - svuota completamente la lista
numeri.clear()
# Ora numeri è []
\end{lstlisting}

\subsubsection{Gestione degli Errori}\label{GestioneErroriListeRemove}

Quando si utilizzano i metodi di rimozione, è importante considerare alcuni possibili errori:

\begin{itemize}
    \item \texttt{remove(x)} genera un errore \texttt{ValueError} se l'elemento \texttt{x} non è presente nella lista
    \item \texttt{pop(i)} e \texttt{del lista[i]} generano un errore \texttt{IndexError} se l'indice \texttt{i} è fuori range
    \item Una volta rimosso un elemento, gli indici degli elementi successivi si spostano per riflettere la nuova posizione
\end{itemize}

Ecco come appaiono questi errori quando si verificano:

\begin{lstlisting}[language=Python]
# Esempio di ValueError con remove()
numeri = [1, 2, 3]
numeri.remove(5)  # Provando a rimuovere un elemento che non esiste
\end{lstlisting}

\begin{tcolorbox}[colback=red!5!white,colframe=red!75!black,fonttitle=\bfseries,title=Output del terminale]
\begin{verbatim}
Traceback (most recent call last):
  File "<stdin>", line 1, in <module>
ValueError: list.remove(x): x not in list
\end{verbatim}
\end{tcolorbox}

\begin{lstlisting}[language=Python]
# Esempio di IndexError con pop()
numeri = [1, 2, 3]
numeri.pop(10)  # Provando ad accedere a un indice che non esiste
\end{lstlisting}

\begin{tcolorbox}[colback=red!5!white,colframe=red!75!black,fonttitle=\bfseries,title=Output del terminale]
\begin{verbatim}
Traceback (most recent call last):
  File "<stdin>", line 1, in <module>
IndexError: pop index out of range
\end{verbatim}
\end{tcolorbox}

\subsubsection{Metodi per Ottenere Informazioni}\label{MetodiInfoListe}

Per ottenere informazioni su una lista e i suoi elementi:

\begin{lstlisting}[language=Python]
colori = ["rosso", "verde", "blu", "verde", "giallo"]

# Funzione len() - restituisce il numero di elementi
lunghezza = len(colori)
print(lunghezza)  # 5

# Metodo count() - conta quante volte un elemento appare nella lista
occorrenze_verde = colori.count("verde")
print(occorrenze_verde)  # 2

# Metodo index() - trova l'indice della prima occorrenza di un elemento
indice_blu = colori.index("blu")
print(indice_blu)  # 2

# E' anche possibile specificare l'intervallo di ricerca per index()
indice_verde_secondo = colori.index("verde", 2)  # Cerca da indice 2 in poi
print(indice_verde_secondo)  # 3
\end{lstlisting}

\subsubsection{Esempi di index() e count()}\label{EsempiInputCountListe}

I metodi \texttt{index()} e \texttt{count()} sono utili per trovare e contare elementi in una lista. Vediamo come funzionano con un esempio visuale:

\begin{lstlisting}[language=Python]
colori = ["rosso", "verde", "blu", "verde", "giallo"]

# Contare le occorrenze di "verde"
occorrenze = colori.count("verde")    # Risultato: 2

# Trovare l'indice della prima occorrenza di "blu"
indice = colori.index("blu")          # Risultato: 2

# Trovare l'indice della seconda occorrenza di "verde"
indice_secondo_verde = colori.index("verde", 2)  # Risultato: 3
                                                 # Comincia a cercare dall'indice 2
\end{lstlisting}





\subsubsection{Metodi per Ordinamento e Inversione}\label{MetodiOrdinamentoInversioneListe}

Per ordinare o invertire gli elementi di una lista:

\begin{lstlisting}[language=Python]
numeri = [3, 1, 4, 1, 5, 9, 2]

# Metodo sort() - ordina la lista in ordine crescente (modifica la lista originale)
numeri.sort()
print(numeri)  # [1, 1, 2, 3, 4, 5, 9]

# Metodo sort() con parametro reverse - ordine decrescente
numeri.sort(reverse=True)
print(numeri)  # [9, 5, 4, 3, 2, 1, 1]

# Funzione sorted() - crea una nuova lista ordinata senza modificare l'originale
parole = ["zebra", "albero", "cane"]
parole_ordinate = sorted(parole)
print(parole)          # ["zebra", "albero", "cane"] (non modificata)
print(parole_ordinate) # ["albero", "cane", "zebra"]

# Metodo reverse() - inverte l'ordine degli elementi
numeri.reverse()
print(numeri)  # [1, 1, 2, 3, 4, 5, 9]
\end{lstlisting}

\begin{nota}
I metodi \texttt{sort()} e \texttt{reverse()} modificano direttamente la lista originale e non restituiscono una nuova lista. Se hai bisogno di mantenere la lista originale, creando una copia, usa la funzione \texttt{sorted()} per l'ordinamento o lo slicing \texttt{[::-1]} per l'inversione.
\end{nota}



\begin{lstlisting}[language=Python]
# Metodo copy() - crea una copia superficiale della lista
originale = [1, 2, 3]
copia = originale.copy()
originale.append(4)
print(copia)  # [1, 2, 3] (la copia non è influenzata)

# Funzione list() - un altro modo per creare una copia
altra_copia = list(originale)
print(altra_copia)  # [1, 2, 3, 4]
\end{lstlisting}

\begin{tcolorbox}[colback=blue!5!white,colframe=blue!75!black,title=Riassunto dei metodi principali delle liste]
\begin{tabularx}{\textwidth}{|l|X|}
\hline
\textbf{Metodo} & \textbf{Descrizione} \\
\hline
\texttt{append(x)} & Aggiunge l'elemento \texttt{x} alla fine della lista \\
\hline
\texttt{insert(i, x)} & Inserisce l'elemento \texttt{x} alla posizione \texttt{i} \\
\hline
\texttt{extend(iterable)} & Aggiunge tutti gli elementi dell'iterabile alla fine della lista \\
\hline
\texttt{remove(x)} & Rimuove la prima occorrenza dell'elemento \texttt{x} \\
\hline
\texttt{pop([i])} & Rimuove e restituisce l'elemento alla posizione \texttt{i} (o l'ultimo se \texttt{i} non è specificato) \\
\hline
\texttt{clear()} & Rimuove tutti gli elementi dalla lista \\
\hline
\texttt{index(x[, start[, end]])} & Restituisce l'indice della prima occorrenza di \texttt{x} (opzionalmente cercando solo dalla posizione \texttt{start} alla \texttt{end}) \\
\hline
\texttt{count(x)} & Restituisce il numero di occorrenze di \texttt{x} nella lista \\
\hline
\texttt{sort([key=None, reverse=False])} & Ordina gli elementi della lista (in-place) \\
\hline
\texttt{reverse()} & Inverte l'ordine degli elementi nella lista (in-place) \\
\hline
\texttt{copy()} & Restituisce una copia superficiale della lista \\
\hline
\end{tabularx}
\end{tcolorbox}
\vspace{0.3cm}


\textbf{Ecco un esempio completo che mostra l'uso di diversi metodi delle liste:}

\begin{lstlisting}[language=Python]
# Creazione di una lista di numeri
numeri = [5, 2, 8, 1, 9, 3]

# Manipolazione della lista
numeri.append(7)           # Aggiunge 7 alla fine: [5, 2, 8, 1, 9, 3, 7]
numeri.insert(0, 0)        # Inserisce 0 all'inizio: [0, 5, 2, 8, 1, 9, 3, 7]
numeri.remove(9)           # Rimuove il 9: [0, 5, 2, 8, 1, 3, 7]
elemento = numeri.pop(3)   # Rimuove e restituisce l'elemento all'indice 3 (8)
                           # numeri ora è [0, 5, 2, 1, 3, 7]

# Informazioni sulla lista
lunghezza = len(numeri)    # 6
occorrenze = numeri.count(5)  # 1
posizione = numeri.index(3)   # 4

# Ordinamento e inversione
numeri.sort()              # [0, 1, 2, 3, 5, 7]
numeri.reverse()           # [7, 5, 3, 2, 1, 0]
\end{lstlisting}