\newpage

\subsection{Esercizi Riassuntivi}\label{EserciziFondamenti}
In questa sezione andremo a svolgere alcuni esercizi per familiarizzare con i concetti affrontati; a fine sezione si troveranno gli esercizi completi con le soluzioni spiegate passo passo.

\subsubsection{Esercizio 1: Implementazione Input ed Output}\label{esercizioFond:1}
Dall'esercizio che abbiamo affrontato nella prima sezione "\nameref{primoProgramma}"\\
- implementare la possibilità di chiedere all'utente la sua età\\
- calcolare l'anno di nascita \\
- stampare il messaggio in modo che comprenda nome anno di nascita.
\begin{lstlisting}
# Il mio primo programma Python
print("Hello, World!")  # Stampa un messaggio a schermo

# Richiesta di input all'utente
nome = input("Come ti chiami? ")
print(f"Ciao {nome}, benvenuto nel mondo di Python!")

# Aggiungi a questo script la possibilita di chiedere all'utente la sua eta

#Calcolare l'anno di nascita approssimativo (anno corrente - eta)

#Stampare un messaggio personalizzato che includa il nome e l'anno di nascita
\end{lstlisting}

\subsubsection{Esercizio 2: Identificazione degli errori}\label{esercizioFond:2}
In questo esercizio viene fornita una serie di variabili la quale denominazione presenta degli errori, identificare gli errori e fornire la denominazione corretta secondo le best practice discusse nella sezione: "\nameref{bestPractice}".
\begin{lstlisting}
1nome = "Paolo"  
nome-cognome = "Mario Rossi"  
class = "Principiante"  

# Di seguito fornisci le tue correzioni
\end{lstlisting}

\subsubsection{Esercizio 3: Operazioni Matematiche}\label{esercizioFond:3}
In questo esercizio dobbiamo dichiarare due variabili numeriche e applicare tutte le possibili operazioni matematiche  (addizione, sottrazione, moltiplicazione, divisione, divisione intera, modulo, potenza) e stampare i relativi risultati.\\
Le sezioni da ripassare per questo esercizio sono:\\
- \nameref{operatoriMatemateci}

\begin{lstlisting}
# Dichiara le 2 variabili
numero1 = # immetti un tuo numero a piacere
numero2 = # immetti un tuo numero a piacere

    # Esegui tutte le operazioni matematiche  (addizione, sottrazione, moltiplicazione, divisione, divisione intera, modulo, potenza)

somma = 
sottrazione = 
divisione = 
moltiplicazione = 
divisione_intera = 
modulo = 
potenza = 

\end{lstlisting}

\subsubsection{Esercizio 4: Operatori logici}\label{esercizioFond:4}
In questo esercizio dovremo capire come funzionano gli operatori logici, l'esercizio fornito ci offre un'espressione logica che dovremo descrivere manualmente per capire il suo significato ed il significato del suo risultato.\\
\textbf{\textit {Bonus: prova differenti versioni dell'esercizio}}

\begin{lstlisting}
a = 5
b = 10
c = 15
risultato = (a < b) and (b <= c) or not (a == c)
print(risultato)

# Spiega perche il risultato del print e' True
\end{lstlisting}

\subsubsection{Esercizio 5: MiniCalcolatrice}\label{esercizioFond:5}
In questo esercizio proveremo a creare una mini calcolatrice, chiederemo all'utente di inserire 2 numeri, dopodiché svolgeremo delle operazioni con i numeri chiesti all'utente.\\
Le sezioni da ripassare per questo esercizio sono le seguenti:\\
- \nameref{InputOutput}\\
- \nameref{operatoriMatemateci}
\vspace{0.3cm}
\begin{lstlisting}
# Chiediamo all'utente il primo numero
primo_numero = 
# Chiediamo il secondo numero
secondo_numero = 

# Esegui le operazioni matematiche
somma =
divisione = 
moltiplicazione = 

# Stampa i risultati delle operazioni
\end{lstlisting}

\begin{tcolorbox}[colback=gray!5!white,colframe=red!75!black,title=Input e Tipi di Dati: Informazioni Importanti]
\begin{itemize}[leftmargin=*,itemsep=0.8em]
    \item \textbf{Comportamento del metodo \texttt{input()}}:
    \begin{itemize}[itemsep=0.2em]
        \item Il metodo \texttt{input()} restituisce \textbf{sempre} una stringa di testo, anche quando l'utente digita numeri
        \item Per una mini-calcolatrice, è necessario convertire questi input in numeri per eseguire calcoli matematici
        \item Se non convertiamo il tipo, l'operazione \texttt{"5" + "3"} produrrà \texttt{"53"} (concatenazione di stringhe) invece di \texttt{8} (somma numerica)
    \end{itemize}
    
    \item \textbf{Conversione dell'input in valori numerici}:
    \begin{itemize}[itemsep=0.2em]
        \item \texttt{int(input("Messaggio: "))} converte l'input in un numero intero (senza decimali)
        \item \texttt{float(input("Messaggio: "))} converte l'input in un numero decimale (con punto)
        \item Esempio pratico: \texttt{numero1 = float(input("Inserisci il primo numero: "))}
    \end{itemize}
    
    \item \textbf{Suggerimento per la mini-calcolatrice}:
    \begin{itemize}[itemsep=0.2em]
        \item Per le operazioni matematiche, è consigliabile usare \texttt{float()} invece di \texttt{int()} per gestire correttamente i numeri decimali
        \item Assicurati che l'utente inserisca effettivamente dei numeri e non del testo
        \item Per la divisione, considera cosa succede se l'utente tenta di dividere per zero
    \end{itemize}
\end{itemize}
\end{tcolorbox}


\subsubsection{Esercizio 6: Mutabilità  e Immutabilità}\label{esercizioFond:6}
In questo esercizio dovremo identificare quale e come le variabili mutano o restano immutate.\\
Descrivi a mano quali variabili mutano e quali no ed il loro perché e come viene allocata la memoria.\\
Le sezioni da ripassare per questo esercizio sono le seguenti:\\
- \nameref{mutableImmutable}\\
- \nameref{variabiliMemoria}
\vspace{0.3cm}
\begin{lstlisting}
# Esempio 1
x = "hello"
y = x
x = "world"
print(y)  # Cosa verra stampato?

# Esempio 2
lista_a = [1, 2, 3]
lista_b = lista_a
lista_a[0] = 99
print(lista_b)  # Cosa verra stampato?
\end{lstlisting}


\subsection{Risoluzione Esercizi sui Fondamentali}
\subsubsection{Risoluzione Esercizio 1: \textit{\nameref{esercizioFond:1}}}

\begin{lstlisting}
# Il mio primo programma Python
print("Hello, World!")  # Stampa un messaggio a schermo

# Richiesta di input all'utente
nome = input("Come ti chiami? ")
print(f"Ciao {nome}, benvenuto nel mondo di Python!")

# Aggiungi a questo script la possibilita di chiedere all'utente la sua eta
eta_utente = int(input("Digita la tua eta:\n"))

#Calcolare l'anno di nascita approssimativo (anno corrente - eta)
anno_corrente = int(input("Digita l'anno corrente"))
anno_nascita = anno_corrente - eta_utente


#Stampare un messaggio personalizzato che includa il nome e l'anno di nascita
print(f"Ciao {nome}, sono nato/a il {anno_nascita}")
\end{lstlisting}

\vspace{0.3cm}

\subsubsection{Risoluzione Esercizio 2: \textit{\nameref{esercizioFond:2}}}

\begin{lstlisting}
1nome = "Paolo"                    # Inizia con un numero
nome-cognome = "Mario Rossi"       # Usa il trattino
class = "Principiante"             # Usa una parola riservata

# Di seguito fornisci le tue correzioni
nome = "Paolo"          
nome_cognome = "Mario Rossi"
classe = "Principiante"
\end{lstlisting}

\vspace{0.3cm}

\subsubsection{Risoluzione Esercizio 3: \textit{\nameref{esercizioFond:3}}}

\begin{lstlisting}
# Dichiara le 2 variabili
numero1 = 12
numero2 = 56

    # Esegui tutte le operazioni matematiche  (addizione, sottrazione, moltiplicazione, divisione, divisione intera, modulo, potenza)

somma = numero1 + numero2
sottrazione = numero1 - numero2
divisione = numero1 / numero2
moltiplicazione = numero1 * numero2
divisione_intera = numero1 // numero2 
modulo = numero1 % numero2
potenza = numero1**numero2


# Stampiamo i risultati 
print(somma, " somma")                              # Output 68
print(sottrazione, " sottrazione")                  # Output -44
print(divisione, " divisione")                      # Output 0.21428571428571427
print(moltiplicazione, " moltiplicazione")          # Output 672
print(divisione_intera, " divisione con intero")    # Output 0
print(modulo, " resto della divisione")             # Output 12
print(potenza, " potenza di numero1 alla numero2")  # Output 2.71 10^60
\end{lstlisting}


\vspace{0.3cm}

\subsubsection{Risoluzione Esercizio 4: \textit{\nameref{esercizioFond:4}}}

\begin{lstlisting}
a = 5
b = 10
c = 15
risultato = (a < b) and (b <= c) or not (a == c)
print(risultato)

# Spiega perche il risultato del print e' True

"""
Per comprendere perche il risultato dell'espressione e True, dobbiamo analizzarla passo per passo seguendo l'ordine di valutazione degli operatori in Python.
Prima valutiamo tutte le espressioni di confronto:
(a < b) significa (5 < 10), che e True
(b <= c) significa (10 <= 15), che e True
(a == c) significa (5 == 15), che e False
Poi applichiamo l'operatore not:
not (a == c) diventa not False, che e True
A questo punto la nostra espressione equivale a:
True and True or True
Ora applichiamo l'operatore and, che ha precedenza rispetto a or:
True and True risulta in True
Infine, applichiamo l'operatore or:
True or True risulta in True
In questo caso specifico, sia la parte a sinistra dell'operatore or ((a < b) and (b <= c)) sia la parte a destra (not (a == c)) sono entrambe True. Di conseguenza, l'intero risultato e True.
E interessante notare che anche se la parte sinistra fosse False, il risultato complessivo sarebbe comunque True perche la parte destra e True e l'operatore or restituisce True se almeno uno dei suoi operandi e True.

"""


\end{lstlisting}



\vspace{0.3cm}

\subsubsection{Risoluzione Esercizio 5: \textit{\nameref{esercizioFond:5}}}

\begin{lstlisting}
# Chiediamo all'utente il primo numero
primo_numero = int(input("Digita un numero intero\n"))
# Chiediamo il secondo numero
secondo_numero = int(input("Digita un secondo numero intero\n"))

# Esegui le operazioni matematiche
somma = primo_numero + secondo_numero
divisione = primo_numero / secondo_numero
moltiplicazione = primo_numero * secondo_numero

# Stampa i risultati delle operazioni
print(f"Somma: {somma}")                        # Output 16
print(f"Divisione: {divisione}")                # Output 3
print(f"Moltiplicazione: {moltiplicazione}")    # Output 48


# Avremmo pututo usare il modulo float al posto di int per avere numeri decimali


\end{lstlisting}



\vspace{0.3cm}

\subsubsection{Risoluzione Esercizio 6: \textit{\nameref{esercizioFond:6}}}

\begin{lstlisting}

# Esempio 1
x = "hello"
y = x
x = "world"
print(y)  # Cosa verra stampato?

# Esempio 2
lista_a = [1, 2, 3]
lista_b = lista_a
lista_a[0] = 99
print(lista_b)  # Cosa verra stampato?

"""

Analisi passo per passo

Creazione della prima variabile:
x = "hello"
In questo passaggio, Python crea un oggetto stringa con valore "hello" in memoria e assegna alla variabile x un riferimento a quest'oggetto. Le stringhe in Python sono tipi immutabili, quindi una volta creato l'oggetto "hello", non potra essere modificato.
Creazione della seconda variabile:
y = x
Qui, Python non crea un nuovo oggetto stringa. Invece, fa puntare la variabile y allo stesso oggetto a cui punta x. Quindi ora sia x che y fanno riferimento allo stesso oggetto stringa "hello" in memoria.
Riassegnazione della prima variabile:
x = "world"
Poiche le stringhe sono immutabili, Python non puo modificare l'oggetto "hello" esistente. Invece, crea un nuovo oggetto stringa con valore "world" e fa puntare x a questo nuovo oggetto. Importante: la variabile y continua a puntare all'oggetto originale "hello".
Stampa del valore:
print(y)
Quando stampiamo y, otteniamo il valore dell'oggetto a cui y fa riferimento, che e ancora "hello".

Risultato
La risposta alla domanda "Cosa verra stampato?" e: hello
Spiegazione del comportamento
Questo esempio illustra perfettamente il comportamento dei tipi immutabili in Python:

Le stringhe sono tipi immutabili, quindi non possono essere modificate dopo la creazione.
Quando assegniamo un valore a una variabile (x = "hello"), creiamo un riferimento a un oggetto.
Quando copiamo una variabile in un'altra (y = x), entrambe le variabili puntano allo stesso oggetto.
Quando riassegniamo un nuovo valore a una variabile (x = "world"), non modifichiamo l'oggetto originale ma creiamo un nuovo oggetto e facciamo puntare la variabile al nuovo oggetto.
La variabile che non e stata riassegnata (y) continua a puntare all'oggetto originale.

Questo comportamento e diverso da quello dei tipi mutabili (come liste o dizionari), dove la modifica di un oggetto attraverso una variabile si rifletterebbe su tutte le variabili che puntano a quell'oggetto.





--------- Rappresentazione Grafica della memoria ---------


                Dopo x = "hello":
                Variabili       Memoria
                ---------       -------
                    x    ---->  "hello"


                Dopo y = x:
                Variabili       Memoria
                ---------       -------
                    x    ---->  "hello"
                    y    ----/


                Dopo x = "world":
                Variabili       Memoria
                ---------       -------
                    x    ---->  "world"
                    y    ---->  "hello
                    
                    
                    
"""

\end{lstlisting}