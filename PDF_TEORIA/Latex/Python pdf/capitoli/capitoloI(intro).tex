\newpage

\section{Introduzione a Python}

Python è un linguaggio di programmazione di alto livello, interpretato, orientato agli oggetti, con una sintassi semplice ed elegante. La sua popolarità è dovuta alla facilità di apprendimento e alla versatilità che lo rende adatto a molteplici ambiti.

\subsection{Caratteristiche Principali}
    \begin{itemize}
        \item  \textbf{Leggibilità}: La sintassi progettata per essere chiara e leggibile
        \item   \textbf{Versatilità}: Adatto a diversi contesti (web, data science, automazione, ecc.)
        \item \textbf{Gestione automatica della memoria}: Non è necessario allocare o deallocare manualmente la memoria
    \end{itemize}

\subsection{Installazione}
Il primo passo da compiere per iniziare a programmare con Python è installarlo sul proprio sistema. Python è disponibile per tutti i principali sistemi operativi: Windows, macOS e Linux.
\subsubsection{Scelta della versione}
Attualmente esistono due versioni principali di Python: Python 2 e Python 3. Si consiglia vivamente di utilizzare Python 3, poiché Python 2 non è più supportato dal gennaio 2020 e molte librerie moderne funzionano esclusivamente con Python 3.
\subsubsection{Download e installazione}
\begin{itemize}
\item \textbf{Windows}:
\begin{enumerate}
\item Visita il sito ufficiale Python (\url{https://www.python.org/downloads/})
\item Scarica l'ultima versione di Python 3 (ad esempio Python 3.11.x)
\item Esegui il file di installazione scaricato
\item \textbf{Importante:} Nella finestra di installazione, seleziona la casella \texttt{Add Python to PATH} prima di procedere
\item Clicca su \texttt{Install Now} per un'installazione standard o \texttt{Customize installation} per opzioni avanzate
\end{enumerate}
\item \textbf{macOS}:
\begin{enumerate}
    \item Visita il sito ufficiale Python (\url{https://www.python.org/downloads/})
    \item Scarica l'ultima versione di Python 3 per macOS
    \item Apri il file .pkg scaricato e segui le istruzioni di installazione
    \item In alternativa, se utilizzi Homebrew, puoi installare Python con il comando \texttt{brew install python3}
\end{enumerate}

\item \textbf{Linux}:
\begin{enumerate}
    \item Molte distribuzioni Linux includono già Python preinstallato. Puoi verificare la versione con \texttt{python3 --version} dal terminale
    \item Su Ubuntu/Debian: \texttt{sudo apt update \&\& sudo apt install python3 python3-pip}
    \item Su Fedora: \texttt{sudo dnf install python3}
    \item Su Arch Linux: \texttt{sudo pacman -S python python-pip}
\end{enumerate}
\end{itemize}
\subsubsection{Verifica dell'installazione}
Dopo l'installazione, è importante verificare che Python sia stato installato correttamente:
\begin{enumerate}
\item Apri il terminale (Command Prompt o PowerShell su Windows, Terminal su macOS/Linux)
\item Digita \texttt{python --version} o \texttt{python3 --version}
\item Dovresti vedere in output la versione di Python installata, ad esempio \texttt{Python 3.11.4}
\end{enumerate}
\subsubsection{Gestione dei pacchetti con pip}
Python include un gestore di pacchetti chiamato pip, che permette di installare facilmente librerie aggiuntive. Per verificare che pip sia installato:
\begin{enumerate}
\item Dal terminale, esegui \texttt{pip --version} o \texttt{pip3 --version}
\item Dovresti vedere la versione di pip installata
\end{enumerate}
Per installare una libreria, usa il comando:
\begin{lstlisting}
pip install nome_libreria
\end{lstlisting}
\subsubsection{Ambienti virtuali}
Per i progetti più complessi, è consigliabile utilizzare ambienti virtuali per mantenere le dipendenze separate. Per creare un ambiente virtuale:
\begin{lstlisting}
Creazione dell'ambiente virtuale
python -m venv mio_ambiente
Attivazione dell'ambiente virtuale
Su Windows
mio_ambiente\Scripts\activate
Su macOS/Linux
source mio_ambiente/bin/activate
\end{lstlisting}
\subsubsection{Opzioni alternative}
Se preferisci non installare Python direttamente sul tuo sistema, puoi considerare:
\begin{itemize}
\item \textbf{Anaconda}: Una distribuzione Python che include molte librerie scientifiche (\url{https://www.anaconda.com/})
\item \textbf{Piattaforme online}:
\begin{itemize}
\item Google Colab: Per notebook Python eseguiti nel browser
\item Replit: Per sviluppo Python online
\item PythonAnywhere: Ambiente Python basato su cloud
\item Deepnote come Replit, ambiente di sviluppo online
\end{itemize}
\end{itemize}

\subsection{Il tuo primo programma Python}\label{primoProgramma}

Ecco un esempio classico per iniziare con Python che include sia l'output che l'input:

\begin{lstlisting}
# Il mio primo programma Python
print("Hello, World!")  # Stampa un messaggio a schermo

# Richiesta di input all'utente
nome = input("Come ti chiami? ")
print(f"Ciao {nome}, benvenuto nel mondo di Python!")
\end{lstlisting}

\begin{nota}
\textbf{Come eseguire il codice Python:}
\begin{enumerate}
    \item \textbf{Da terminale}: Salva il codice in un file con estensione .py (es. primo\_programma.py) e eseguilo con il comando \texttt{python primo\_programma.py}
    \item \textbf{Da IDE}: Usa un ambiente di sviluppo come PyCharm, Visual Studio Code o IDLE (incluso nell'installazione di Python)
    \item \textbf{Da notebook}: Jupyter Notebook permette di eseguire il codice in celle interattive
    \item \textbf{Online}: Servizi come replit.com o Google Colab permettono di scrivere ed eseguire codice Python senza installazioni
\end{enumerate}
\end{nota}
