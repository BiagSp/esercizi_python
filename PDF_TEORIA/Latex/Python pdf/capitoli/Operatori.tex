\clearpage
\subsection{Operatori}\label{Operatori}
In questa sottosezione andremo a visionare i principali Operatori presenti su Python, possiamo considerarli come "strumenti" che permettono di manipolare dati, eseguire calcoli e prendere decisioni all'interno del codice. Gli opreatori sono essenzialmente simboli speciale che indicano al computer di eseguire operazioni specifiche sui valori (o operandi) a cui sono applicati.

Alcuni di essi possono risultare alquanto ambigui per la loro essenza e semplicità ma rappresentano un valore chiave per la comprensione dell'azione che bisogna eseguire.


\begin{itemize}
    \item \textbf{Operatori aritmetici}: permettono di eseguire calcoli matematici sui numeri
    \item \textbf{Operatori di confronto}: consentono di comparare valori e restituiscono risultati booleani (vero o falso)
    \item \textbf{Operatori logici}: permettono di combinare condizioni booleane e sono essenziali nelle strutture decisionali
\end{itemize}


Accanto agli operatori, le funzionalità di input e output rappresentano le "porte" attraverso cui un programma comunica con l'utente o con l'ambiente esterno. Attraverso queste funzioni, i programmi possono acquisire dati, presentare risultati e interagire con chi li utilizza.

Gli esempi che seguono illustrano la sintassi e l'uso pratico di ciascun tipo di operatore e delle funzioni di input/output di base in Python.


\subsubsection{Operatori aritmetici}\label{operatoriMatemateci}
\begin{lstlisting}
a = 10
b = 3

somma = a + b          # 13
differenza = a - b     # 7
prodotto = a * b       # 30
divisione = a / b      # 3.3333... (divisione che restituisce float)
div_intera = a // b    # 3 (divisione intera)
modulo = a % b         # 1 (resto della divisione)
potenza = a ** b       # 1000 (a elevato alla b)
\end{lstlisting}

\subsubsection{Operatori di confronto}
\begin{lstlisting}
a == b    # Uguaglianza (False)
a != b    # Disuguaglianza (True)
a > b     # Maggiore (True)
a < b     # Minore (False)
a >= b    # Maggiore o uguale (True)
a <= b    # Minore o uguale (False)
\end{lstlisting}

\subsubsection{Operatori logici}\label{opLogici}
\begin{lstlisting}
x = True
y = False

x and y    # False (AND logico)
x or y     # True (OR logico)
not x      # False (NOT logico)
\end{lstlisting}


\subsection{Input e Output}\label{InputOutput}
\begin{lstlisting}
# Output sulla console
print("Ciao, mondo!")

# Input da tastiera
nome = input("Come ti chiami? ")
print(f"Ciao, {nome}!")

# Formattazione delle stringhe
eta = 25
messaggio = f"Ho {eta} anni."
print(messaggio)
\end{lstlisting}