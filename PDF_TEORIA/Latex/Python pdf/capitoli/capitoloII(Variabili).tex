


\section{Variabili e Tipi di Dati}

\subsection*{Introduzione}
In questa sezione esploreremo il cuore della programmazione Python: variabili e tipi di dati. Vedremo come Python gestisce diversi tipi di informazioni, come memorizzarle in variabili, e come manipolarle attraverso varie operazioni. Comprenderemo le regole per denominare le variabili, come Python interpreta i diversi tipi di dati, e come interagiscono tra loro. 

In Python, le variabili sono contenitori per memorizzare valori di dati. A differenza di altri linguaggi di programmazione, Python non richiede di dichiarare esplicitamente il tipo di una variabile prima di usarla. Il tipo viene determinato automaticamente in base al valore assegnato.

\subsection{Dichiarazione e Assegnazione delle Variabili}\label{bestPractice}
\begin{lstlisting}
# Dichiarazione e assegnazione di variabili
nome = "Mario"      # Una stringa
eta = 30            # Un intero
altezza = 1.75      # Float
e_studente = False  # Booleano
\end{lstlisting}


\subsection{Regole e best practice per denominare le variabili}

In Python, i nomi delle variabili devono seguire regole precise e alcune convenzioni che rendono il codice più leggibile e manutenibile.

\begin{tcolorbox}[colback=blue!5!white,colframe=blue!75!black,title=Regole sintattiche]
\begin{itemize}[leftmargin=*,itemsep=0.5em]
    \item \textbf{Caratteri consentiti}: Lettere (a-z, A-Z), numeri (0-9) e underscore (\_)
    \item \textbf{Primo carattere}: Deve essere una lettera o un underscore, mai un numero
    \item \textbf{Case sensitivity}: \texttt{nome}, \texttt{Nome} e \texttt{NOME} sono tre variabili diverse
    \item \textbf{Parole riservate}: Non è possibile usare parole chiave di Python (come \texttt{if}, \texttt{for}, \texttt{while})
\end{itemize}
\end{tcolorbox}

\begin{tcolorbox}[colback=green!5!white,colframe=green!75!black,title=Convenzioni di stile (PEP 8)]
\begin{itemize}[leftmargin=*,itemsep=0.5em]
    \item \textbf{snake\_case}: Per variabili e funzioni (es. \texttt{nome\_utente})
    \item \textbf{SCREAMING\_SNAKE\_CASE}: Per costanti (es. \texttt{MAX\_TENTATIVI})
    \item \textbf{CamelCase}: Per classi (es. \texttt{PersonaUtente})
    \item \textbf{Nomi significativi}: Evitare sigle come \texttt{x}, \texttt{y}, preferire nomi descrittivi
    \item \textbf{Underscore iniziale}: \texttt{\_variabile} suggerisce uso interno o privato
\end{itemize}
\end{tcolorbox}

\begin{lstlisting}
# Esempi di nomi validi
nome = "Alice"
_eta = 25
nome_completo = "Alice Rossi"
variabile1 = 42
COSTANTE = 3.14159

# Esempi di nomi non validi
# 1variabile = 10    # Inizia con un numero (errore di sintassi)
# for = "ciclo"      # E una parola riservata (errore di sintassi)
# nome-utente = "Mario"  # Il trattino non e' consentito
\end{lstlisting}

\vspace{0.5cm}


\clearpage
\section{Metodi principali per i tipi di dati Python}

\subsection*{Cos'è un metodo in Python?}

In Python, i \textbf{metodi} sono funzioni speciali associate a oggetti di specifici tipi di dati. A differenza delle funzioni standard che accettano dati come argomenti, i metodi sono "collegati" agli oggetti stessi attraverso la notazione punto (\texttt{oggetto.metodo()}).

Ogni tipo di dato in Python possiede metodi specifici progettati per operazioni comuni e naturali su quel tipo. Questa caratteristica è parte fondamentale del paradigma della programmazione orientata agli oggetti che Python supporta.


\begin{lstlisting}[style=pythonstyle]
# Funzione standard
len("Python")  # 6

# Metodo
"Python".upper()  # "PYTHON"
\end{lstlisting}


\subsection*{Metodi e mutabilità}\label{MutabilitàImmutabilità}

Un concetto cruciale per comprendere i metodi in Python è la \textbf{mutabilità}. I tipi di dati in Python si dividono in due categorie:

\begin{itemize}
    \item \textbf{Tipi immutabili}: una volta creati, non possono essere modificati (stringhe, tuple, numeri)
    \item \textbf{Tipi mutabili}: possono essere modificati dopo la creazione (liste, dizionari, set)
\end{itemize}

Questa distinzione influenza profondamente il comportamento dei metodi:

\begin{tcolorbox}[colback=blue!5!white,colframe=blue!75!black,title=Comportamento dei Metodi]
\begin{tabularx}{\textwidth}{|l|X|}
\hline
\textbf{Tipi immutabili} & I metodi \textbf{restituiscono sempre} un nuovo oggetto, lasciando l'originale invariato \\
\hline
\textbf{Tipi mutabili} & I metodi spesso \textbf{modificano direttamente} l'oggetto (operazioni in-place) \\
\hline
\end{tabularx}
\end{tcolorbox}

\subsection*{Metodi delle stringhe}

Le stringhe (\texttt{str}) sono sequenze immutabili di caratteri, quindi i loro metodi restituiscono sempre nuove stringhe. Questi metodi consentono di trasformare e analizzare testo in modi potenti e flessibili.

La potenza dei metodi per stringhe si apprezza particolarmente nell'elaborazione del testo, dove possiamo concatenare più operazioni in sequenza:


\begin{lstlisting}[style=pythonstyle]
# Pulizia e parsing di dati testuali
testo_grezzo = "  python, java, C++  "
linguaggi = testo_grezzo.strip().lower().split(", ")
print(linguaggi)  # ['python', 'java', 'c++']
\end{lstlisting}




\subsection*{Principi guida per l'uso dei metodi}

Quando lavori con i metodi in Python, tieni a mente questi principi:

\begin{enumerate}
    \item \textbf{Verifica la mutabilità}: prima di usare un metodo, capisci se stai lavorando con un tipo mutabile o immutabile
    \item \textbf{Controlla il valore restituito}: alcuni metodi restituiscono un nuovo oggetto, altri modificano l'oggetto esistente
    \item \textbf{Leggi la documentazione}: quando sei in dubbio, consulta la documentazione ufficiale
\end{enumerate}

\begin{nota}
La documentazione ufficiale di Python (\url{https://docs.python.org/3/library/stdtypes.html}) contiene l'elenco completo di tutti i metodi disponibili per ciascun tipo di dato, con spiegazioni dettagliate ed esempi.
\end{nota}

Di seguito vengono riportate le tabelle con i principali metodi per la manipolazione delle stringhe in Python, si tratta di metodi preinstallati in Python:


% Stringhe (str)
\begin{table}[H]
\centering
\caption{Metodi essenziali per le stringhe (\texttt{str})}
\begin{tabularx}{\textwidth}{|l|X|l|}
\hline
\textbf{Metodo} & \textbf{Descrizione} & \textbf{Esempio} \\
\hline
\texttt{upper()} & Converte in maiuscolo & \texttt{"python".upper() → "PYTHON"} \\
\hline
\texttt{lower()} & Converte in minuscolo & \texttt{"Python".lower() → "python"} \\
\hline
\texttt{strip()} & Rimuove spazi agli estremi & \texttt{"  Python  ".strip() → "Python"} \\
\hline
\texttt{replace(old, new)} & Sostituisce le occorrenze & \texttt{"Hello".replace("l", "x") → "Hexxo"} \\
\hline
\texttt{split(sep)} & Divide in una lista & \texttt{"a,b,c".split(",") → ["a", "b", "c"]} \\
\hline
\texttt{join(iterable)} & Unisce una lista in stringa & \texttt{"-".join(["a", "b"]) → "a-b"} \\
\hline
\texttt{find(sub)} & Trova posizione sottostringa & \texttt{"Python".find("th") → 2} \\
\hline
\end{tabularx}
\end{table}



