\clearpage

\subsection{Esercizi Riassuntivi}
Come per la sezione sulle \textit{\hyperref[EserciziFondamenti]{variabili}}, anche qui andiamo a svolgere alcuni esercizi per familiarizzare con i concetti affrontati, anche qui a fine sezione si troveranno gli esercizi svolti con le soluzioni spiegate passo passo.

\subsubsection{Esercizio 1: Concetti Base e Indicizzazione}\label{Esercizio1Liste}
In questo esercizio andremo ad approfondire il concetto di indicizzazione, vedi capitolo \nameref{IndicizzazioneListe}
    \newline
\textbf{Traccia gli Indici:} Data la lista \textit{colori = ["rosso", "verde", "blu", "giallo", "viola"]}

-   Scrivi l'indice positivo e negativo per accedere a ciascun elemento

-   Qual è il valore di colori[1]? E di colori[-2]

-   Se volessi accedere a \textit{"verde"} usando un indice negativo, quale sarebbe?

\vspace{0,5cm}
\begin{lstlisting}
#Dichiariamo la lista colori
colori = ["rosso", "verde", "blu", "giallo", "viola"]

#Inserisci gli indici positivi e negativi di tutti gli elementi della lista

#Valore di colori[1]

#Valore di colori[-2]

#A quale indice corrisponde "verde" e quale sarebbe il suo corrispettivo indice negativo?

\end{lstlisting}

\subsubsection{Esercizio 2: Operazioni Base sulle Liste}\label{Esercizio2Liste}
\title{\textbf{Modifica gli Elementi}}
inizializziamo una lista punteggi.

\begin{lstlisting}
punteggi = [85, 92, 78, 90, 88]
# - Sostituisci il terzo elemento con 95

# - Aggiungi 100 alla fine della lista

# - Inserisci 82 nella seconda posizione

\end{lstlisting}

\title{\textbf{Concatenazione e Ripetizione}}
\begin{lstlisting}
# Scrivi il risultato di queste operazioni:

lista1 = [1, 2, 3]
lista2 = [4, 5]
risultato1 = lista1 + lista2
risultato2 = lista1 * 3
\end{lstlisting}

Per questo esercizio, ripassare le sezioni:
\begin{itemize}
    \item \nameref{MetodiListe}
    \item \nameref{MetodiListeRemove}
\end{itemize}



\subsubsection{Esercizio 3: Slicing delle Liste}\label{Esercizio3Liste}
\title{\textbf{Estrazioni con Slicing}}

\begin{lstlisting}
# Data una lista
lettere = ["a", "b", "c", "d", "e", "f", "g"]

# - Estrarre i primi due elementi

# - Estrai gli ultimi due elementi

# - Estrai gli elementi dallaa posizione 2 alla 5 (inclusa)

# - Estrai tutti gli elementi tranne il primo e l'ultimo
\end{lstlisting}

\title{\textbf{Slicing Avanzato}}
\begin{lstlisting}
# Prevedi il risultato di questo codice:

sequenza = [10, 20, 30, 40, 50, 60, 70]
parte1 = sequenza[1:5]
parte2 = sequenza[::2]
parte3 = sequenza[5:1:-1]
\end{lstlisting}


Per questo esercizio, ripassare le sezioni:
\begin{itemize}
    \item \nameref{SlicingListe}
\end{itemize}

\subsubsection{Esercizio 4: Metodi delle Liste}\label{Esercizio4Liste}
\title{\textbf{Applicazione dei metodi base}}

\begin{lstlisting}
# Data una lista
frutta = ["mela", "banana", "ciliegia"]

# - Usa il metodo appropriato per aggiungere "arancia" alla fine
    
# - Usa il metodo corretto per trovare la posizione di "banana"

# - Usa un metodo per contare quante volte appare "mela"

# - Usa un metodo per rimuovere "ciliegia"

\end{lstlisting}


\title{\textbf{Ordinamento e Inversione:}}


\begin{lstlisting}
# Data una lista di numeri
numeri = [42, 8, 16, 23, 4, 15]

# - Ordina la lista in modo crescente

# - Inverti l'ordine degli elementi

# - Qual è la differenza tra usare il metodo sort() e la funzione sorted()
\end{lstlisting}






Per questo esercizio ripassare le sezioni:
\begin{itemize}
    \item \nameref{MetodiOrdinamentoInversioneListe}
    \item \nameref{EsempiInputCountListe}
    \item  \nameref{AccessoListe}
    \item  \nameref{MetodiInfoListe}
\end{itemize}


\subsubsection{Esercizio 5: Comportamento in Memoria delle Liste}\label{Esercizio5Liste}
\title{\textbf{Riferimenti e Copie}}


\begin{lstlisting}
# Analizza e predici l'Output

a = [1, 2, 3]
b = a
b[0] = 10
print(a)

\end{lstlisting}


\title{\textbf{Creazione di Copie Indipendenti:}}
\begin{lstlisting}
# Quali tra questi metodi creanno una copia indipendente?

lista2 = lista1

lista2 = lista1[:]

lista2 = list(lista1)

lista2 = lista1.copy()

\end{lstlisting}


\title{\textbf{Identità vs Uguaglianza:}}
\begin{lstlisting}
# Predici il risultato di questi confronti:

lista1 = [1, 2, 3]
lista2 = [1, 2, 3]
lista3 = lista1

print(lista1 == lista2)
print(lista1 is lista2)
print(lista1 is lista3)

\end{lstlisting}



Per questo esercizio ripassare le sezioni:
\begin{itemize}
    \item \nameref{ComportamentoInMemoriaListe}
\end{itemize}

\subsubsection{Esercizio 6: Analisi di Codice e Risoluzione Problemi:}\label{Esercizio6Liste}
\title{\textbf{Trova l'errore}}


\begin{lstlisting}
# Indentifica e correggi l'errore in questo codice:

colori = ["rosso", "verde", "blu"]
colori[3] = "giallo"  # Aggiungi giallo alla lista
print(colori[4])      # Stampa il quinto elemento

\end{lstlisting}


\title{\textbf{Tracciamento della Memoria:}}
\begin{lstlisting}
# Dissegna un diagramma di memoria che mostri lo stato delle varaibili dopo ogni istruzione

lista1 = [10, 20]
lista2 = lista1
lista1.append(30)
lista3 = lista1[:]
lista3[0] = 99

\end{lstlisting}


\title{\textbf{Test di Comprensione}}
\begin{lstlisting}
# Cosa Stampa questo codice

a = [1, 2, 3]
b = [a, a]
b[0][1] = 10
print(b)
print(a)

\end{lstlisting}



Per questo esercizio ripassare le sezioni:
\begin{itemize}
    \item \nameref{ComportamentoInMemoriaListe}
\end{itemize}


\subsection{Risoluzione Esercizi Sulle Liste}

\subsubsection{Risoluzione Esercizio 1: \nameref{Esercizio1Liste}}

\begin{lstlisting}
# Data la lista colori = ["rosso", "verde", "blu", "giallo", "viola"]

#Inserisci gli indici positivi e negativi di tutti gli elementi della lista
    #Per risolvere questo quesito potremmo prima vedere quanto è lunga la lista
    print(len(colori))
    
    #In questo modo conosceremo quanti elementi sono presenti nella lista e potremo numerarli in entrambi i versi

    # Formula: indice_negativo = -lunghezza + indice_positivo
    
    # "rosso": indice positivo 0, indice negativo = -5 + 0 = -5
    # "verde": indice positivo 1, indice negativo = -5 + 1 = -4
    # "blu": indice positivo 2, indice negativo = -5 + 2 = -3
    # "giallo": indice positivo 3, indice negativo = -5 + 3 = -2
    # "viola": indice positivo 4, indice negativo = -5 + 4 = -1
        
#Valore di colori[1]
    #Dopo aver numerato gli elementi della lista ora possiamo ricavare i valori all'indice richiesto
#    - Verde
#Valore di colori[-2]
#    - Giallo
#A quale indice corrisponde "verde" e quale sarebbe il suo corrispettivo indice negativo?
#Ricaviamolo dalla numerazione svolta prima
#"verde"  - lunghezza 5 + indice positivo 1 = 1 -5 = -4
\end{lstlisting}


\subsubsection{Risoluzioe Esercizio 2: \nameref{Esercizio2Liste}}

\begin{lstlisting}
punteggi = [85, 92, 78, 90, 88]
# - Sostituisci il terzo elemento con 95

    # Prima di sostituire il numero troviamolo, stampiamo la lunghezza della lista cosi da facilitare la numerazione degli elementi.
    print(len(punteggi))
    # Ora numeramio gli elementi
    # b = 80 (30 + 50 = 80, il terzo elemento più l'ultimo)
    # a = 10 (il primo elemento)
    # c = 30 (il terzo elemento dalla fine)

# Ora possiamo sostituire il terzo elemento con 95
    punteggi[2] = 95
    
# - Aggiungi 100 alla fine della lista
    # Qui possiamo agire in diversi modi: 
    
    #Concatenazione con una lista contenente il nuovo elemento:
        punteggi = punteggi + [100]
    
    #Operatore di aggiunta in-place:
        punteggi += [100]
    
    #Metodo extend con una lista singola:
        punteggi.extend([100])
    
    #Slice assignment alla fine della lista:
        punteggi[len(punteggi):] = [100]
    
    #Insert all'indice finale:
        punteggi.insert(len(punteggi), 100)
    
    # - Inserisci 82 nella seconda posizione


# === SECONDA PARTE ===

# Scrivi il risultato di queste operazioni:

lista1 = [1, 2, 3]
lista2 = [4, 5]
risultato1 = lista1 + lista2
risultato2 = lista1 * 3

    # risultato1 = [1, 2, 3, 4, 5] (concatenazione delle due liste)
    # risultato2 = [1, 2, 3, 1, 2, 3, 1, 2, 3] (ripetizione della prima lista 3 volte)
\end{lstlisting}


\subsubsection{Risoluzione Esercizio 3: \nameref{Esercizio3Liste}}

\begin{lstlisting}
# Data una lista
lettere = ["a", "b", "c", "d", "e", "f", "g"]

# Primi tre elementi
primi_tre = lettere[:3]  # ["a", "b", "c"]

# Ultimi due elementi
ultimi_due = lettere[-2:]  # ["f", "g"]

# Elementi dalla posizione 2 alla 5 (inclusa)
dalla_2_alla_5 = lettere[2:6]  # ["c", "d", "e", "f"]

# Tutti tranne primo e ultimo
senza_primo_ultimo = lettere[1:-1]  # ["b", "c", "d", "e", "f"]

# === SECONDA PARTE ===

Prevedi il risultato di questo codice:

sequenza = [10, 20, 30, 40, 50, 60, 70]
parte1 = sequenza[1:5]
parte2 = sequenza[::2]
parte3 = sequenza[5:1:-1]

#parte1 = [20, 30, 40, 50] (elementi dall'indice 1 all'indice 4 incluso)
#parte2 = [10, 30, 50, 70] (ogni elemento a passo 2, partendo dall'inizio)
#parte3 = [60, 50, 40, 30] (dall'indice 5 all'indice 2 incluso, in ordine inverso)
\end{lstlisting}

\subsubsection{Risoluzione Esercizio 4: \nameref{Esercizio4Liste}}

\begin{lstlisting}
# Data una lista
frutta = ["mela", "banana", "ciliegia"]

# - Usa il metodo appropriato per aggiungere "arancia" alla fine
    frutta.append("arancia")
    print(frutta)  # ["mela", "banana", "ciliegia", "arancia"]
    
# - Usa il metodo corretto per trovare la posizione di "banana"
    posizione_banana = frutta.index("banana")  # 1
    
# - Usa un metodo per contare quante volte appare "mela"
    occorrenze_mela = frutta.count("mela")  # 1
    
# - Usa un metodo per rimuovere "ciliegia"
    frutta.remove("ciliegia")
    print(frutta)  # ["mela", "banana", "arancia"]


# === SECONDA PARTE ===


# Data una lista di numeri
numeri = [42, 8, 16, 23, 4, 15]

# - Ordina la lista in modo crescente
    numeri.sort()
    print(numeri)  # [4, 8, 15, 16, 23, 42]
    
# - Inverti l'ordine degli elementi
    numeri.reverse()
    print(numeri)  # [42, 23, 16, 15, 8, 4]
    
# - Qual è la differenza tra usare il metodo sort() e la funzione sorted()
    # Differenza tra sort() e sorted():
    # - sort() modifica la lista originale
    # - sorted() crea una nuova lista ordinata lasciando l'originale intatta

originale = [3, 1, 2]
nuova = sorted(originale)
print(originale)  # [3, 1, 2] rimane invariata
print(nuova)      # [1, 2, 3] nuova lista ordinata
\end{lstlisting}


\subsubsection{Risoluzione Esercizio 5: \nameref{Esercizio5Liste}}

\begin{lstlisting}
# Analizza e predici l'Output

a = [1, 2, 3]
b = a
b[0] = 10
print(a)

    # Soluzione: L'output sarà [10, 2, 3] 
    # Spiegazione: b è un riferimento alla stessa lista di a. Modificando b, si  # modifica anche a perchè entrambe le variabili puntano allo stesso oggetto # in memoria.


# === SECONDA PARTTE ===


# Predici il risultato di questi confronti:

lista1 = [1, 2, 3]
lista2 = [1, 2, 3]
lista3 = lista1

print(lista1 == lista2)
print(lista1 is lista2)
print(lista1 is lista3)

# lista1 == lista2  -> True (confronta i contenui che sono uguali)
# lista1 is lista2 -> False (si riferiscono a oggetti diversi in memoria)
# lista1 is lista3  -> True (si riferiscono allo stesso oggetto in memoria)


lista1 == lista2 
#True (confronta i contenuti, che sono uguali)
lista1 is lista2 
#False (si riferiscono a oggetti diversi in memoria)
lista1 is lista3 
#True (si riferiscono allo stesso oggetto in memoria)
\end{lstlisting}


\subsubsection{Risoluzione Esercizio 6: \nameref{Esercizio6Liste}}

\begin{lstlisting}
# Indentifica e correggi l'errore in questo codice:

colori = ["rosso", "verde", "blu"]
colori[3] = "giallo"  # Aggiungi giallo alla lista
print(colori[4])      # Stampa il quinto elemento

#Non si puo usare colori[3] = "giallo" perchè l'indice 3 è fuori dal range (la #lista ha indici 0, 1, 2).
#Anche dopo aver aggiunto "giallo", colori[4] sarebbe fuori range.

# Correzione:
colori = ["rosso", "verde", "blu"]
colori.append("giallo")  # Corretto: aggiunge "giallo" alla fine
print(colori[3])         # Corretto: stampa il quarto elemento (giallo)



# === SECONDA PARTE === 

# Dissegna un diagramma di memoria che mostri lo stato delle varaibili dopo ogni istruzione


lista1 = [10, 20]
lista2 = lista1
lista1.append(30)
lista3 = lista1[:]
lista3[0] = 99


#lista1 = [10, 20]:

#lista1 -> [10, 20]


lista2 = lista1:

#lista1 -> [10, 20]
#lista2 -> [10, 20] (stesso oggetto di lista1)


lista1.append(30):

#lista1 -> [10, 20, 30]
#lista2 -> [10, 20, 30] (stesso oggetto di lista1)


lista3 = lista1[:]:

#lista1 -> [10, 20, 30]
#lista2 -> [10, 20, 30] (stesso oggetto di lista1)
#lista3 -> [10, 20, 30] (nuovo oggetto, copia di lista1)


lista3[0] = 99:

#lista1 -> [10, 20, 30]
#lista2 -> [10, 20, 30] (stesso oggetto di lista1)
#lista3 -> [99, 20, 30]


# === TERZA PARTE ===

# Cosa stampa questo codice
a = [1, 2, 3]
b = [a, a]
b[0][1] = 10
print(b)
print(a)


print(b): [[1, 10, 3], [1, 10, 3]]
print(a): [1, 10, 3]

#Spiegazione:

#a è una lista [1, 2, 3]
#b è una lista che contiene due riferimenti alla stessa lista a: [[1, 2, 3], [1, #2, 3]]
#b[0][1] = 10 modifica il secondo elemento della prima sottolista di b (che è #a), #quindi ora a è [1, 10, 3]
#Poichè entrambi gli elementi di b sono riferimenti alla stessa lista a, la #modifica si riflette in entrambiprint(b): [[1, 10, 3], [1, 10, 3]]
#print(a): [1, 10, 3]
\end{lstlisting}

\newpage